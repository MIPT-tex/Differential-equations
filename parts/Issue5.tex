\subsection{Матричная экспонента, ее свойства и применение к решению нормальных линейных систем}

\subsubsection{Матричная экспонента}
Необходимо решить ОЛДУ вида:

\begin{equation}
  	\frac{d\overrightarrow{x}}{dt} = A\overrightarrow{x},\ \overrightarrow{x}(t_0) = \overrightarrow{x_0},
  	\label{Issue5_1}
\end{equation}

Матрица $A(t) = ||a_j^i||,\ a_j^i \in \mathbf{R},\ i, j = 1,\dots,\ n$. При доказательстве теоремы существования и единственности была использована следующая итерационная процедура:

\begin{equation*}
	\overrightarrow{x^{0}} = \overrightarrow{x_{0}}, \; \overrightarrow{x^{1}}(t) = \overrightarrow{x_{0}} + \int \limits_{t_0}^{t} A \overrightarrow{x_{0}} ds, \; \dots, \overrightarrow{x^{n}}(t) = \overrightarrow{x_{0}} + \int \limits_{t_0}^{t} A \overrightarrow{x^{n - 1}} ds
\end{equation*}

Отсюда можно получить:

\begin{equation*}
\begin{gathered}
         \overrightarrow{x_{0}} = E\overrightarrow{x_0},\ \overrightarrow{x_1} = E\overrightarrow{x_0} + \frac{t-t_0}{1!}A\overrightarrow{x_0} = \left(E + \frac{t-t_0}{1!}A\right) \overrightarrow{x_0}, \\  
         \overrightarrow{x_n} = \left(E + \frac{t-t_0}{1!}A +\ \dots\ + \frac{(t-t_0)^n}{n!}A^n\right)\overrightarrow{x_0},	 
\end{gathered}
\end{equation*}

При доказательстве теоремы существования и единственности Коши было получено, что решением является

\begin{equation*}
   	\overrightarrow{x} = \left(E + \frac{t-t_0}{1!}A +\ \dots\ + \frac{(t-t_0)^n}{n!}A^n+\ \dots \right)\overrightarrow{x_0} = \left(\sum\limits_{n = 0}^{\infty} \frac{(t-t_0)^n}{n!}A^n\right)\overrightarrow{x_0},	 
\end{equation*}

при условии, что $A^0 = E.$

\begin{definition}
	Матричной экспонентой называют следующий степенной ряд:
	\begin{equation*}
		e^{(t-t_0)A} = E + \frac{t-t_0}{1!}A +\ \dots\ + \frac{(t-t_0)^n}{n!}A^n+\ \dots\ = \sum\limits_{n = 0}^{\infty} \frac{(t-t_0)^n}{n!}A^n.
	\end{equation*}
\end{definition}

\subsubsection{Свойства матричной экспоненты}

Матричная экпонента это квадратная матрица, по размерам аналогична матрице $A$, и каждый элемент этой матрицы представляет из себя степенной ряд с радиусом сходимости $+\infty$.

\begin{enumerate}
	\item $e^{(t_1+t_2)A} = e^{t_1A}e^{t_2A} \Rightarrow e^{t_1A}e^{t_2A} = e^{t_2A}e^{t_1A}$ (коммутативность).
		
	\item $ e^{0 A} = E.$	
	
	\item $\left(e^{tA}\right)^{-1} = e^{-tA}.$
		
	\item $(e^{tA})^{'} = A e^{tA} = e^{tA}A.$

\end{enumerate}

\begin{proof}

Так как квадратные матрицы составляют определенное кольцо, то \\

\begin{enumerate}

	\item Рассматриваем $(\ref{Issue5_1})$, если $\overrightarrow{x}(t)$ -- решение этого ДУ, то $\overrightarrow{x}(t+t_0)$ тоже решение этого ДУ: пусть $u = t + t_0$, тогда

	\[ \frac{d\overrightarrow{x}(t+t_0)}{dt} = \frac{d\overrightarrow{x}}{du}\frac{du}{dt} = \frac{d\overrightarrow{x}}{du} = A\overrightarrow{x}(u) = A\overrightarrow{x}(t+t_0).\]

	Рассмотрим $(\ref{Issue5_1})$ с задачей Коши $\overrightarrow{x}(0) = \overrightarrow{x_0}$: система имеет решение

	\[ \overrightarrow{x}(t) =  e^{tA}\overrightarrow{x_0},\]
	\[ \overrightarrow{x}(t+t_0) = e^{(t+t_0)}\overrightarrow{x_0}\text{ -- решение }\frac{d\overrightarrow{x}}{dt} = A\overrightarrow{x}. \]

	Рассмотрим тогда то же самое уравнение для функции $\overrightarrow{z}(t)$:

	\[ \frac{d\overrightarrow{z}}{dt} = A\overrightarrow{z}, \text{ с задачей Коши } \overrightarrow{z}(0) = e^{t_1A}\overrightarrow{x_0} \Rightarrow \overrightarrow{z}(t) = e^{tA} (e^{t_1A}\overrightarrow{x_0}) = (e^{tA}e^{t_1A})\overrightarrow{x_0}.\]

	Получаем:

	\[ \overrightarrow{x}(t_1) = e^{t_1A}\overrightarrow{x_0} = \overrightarrow{z}(0),\]

	из основной теоремы следует, что $\overrightarrow{x}(t+t_1) = \overrightarrow{z}(t)\ \forall t$.

	Тогда и получается основная формула:

	\[ \overrightarrow{x}(t+t_1) = e^{(t+t_1)A}\overrightarrow{x_0} = (e^{tA}e^{t_1A})\overrightarrow{x_0} = \overrightarrow{z}(t)\]

	\item $ e^{tA} = E + \frac{t-t_0}{1!}A +\ \dots\ + \frac{(t-t_0)^n}{n!}A^n+\ \dots $, если $t = 0$ :
	\[ e^{0A} = E + 0 + \dots = E\]

	\item $E = e^{0A} = e^{(t-t)A} = e^{tA}e^{-tA} = E \Rightarrow \left(e^{tA}\right)^{-1} = e^{-tA}.$

	\item Берем представление матричной экспоненты в виде степенного ряда, который можно дифференцировать, тогда получаем:

	\[ (e^{tA})^{'} = A + tA^2 +\ \dots\ +\frac{t^{n-1}}{(n-1)!} A^n +\ \dots = A\left(E + tA +\ \dots\ + \frac{t^{n-1}}{(n-1)!}A^{n-1}\right),\]

	\[ (e^{tA})^{'} = Ae^{tA} = e^{tA}A.\]

\end{enumerate}

\end{proof}

\textbf{Примечание.} Формула $e^{t(A+B)} = e^{tA}e^{tB}$ не имеет места, кроме случая, если $AB = BA$ (т.е. матрицы коммутативны).

\subsubsection{Применение к решению нормальных линейных систем}

\begin{theorem}

Пусть $S$ -- матрица перехода от исходного базиса к новому базису. Тогда в новой базисе $\overline{A} = S^{-1}AS$, или $A = S\overline{A}S^{-1}$. И главное:

\[e^{t\overline{A}} = S^{-1}e^{tA}S.\]

\end{theorem}

\begin{proof}
	\begin{equation*}
		\begin{gathered}
 e^{t\overline{A}} = \left(E + t\overline{A} +\ \dots\ + \frac{t^{n}}{n!}\overline{A}^{n} \dots \right) = \left(E + tS^{-1}AS +\ \dots\ + \frac{t^{n}}{n!}(S^{-1}AS)^{n}\right), \\
 (S^{-1}AS)^n = S^{-1}A^nS,\ SES^{-1} = SS^{-1} = E \\
 e^{t\overline{A}} = S^{-1}e^{tA}S.
		\end{gathered}
	\end{equation*}
\end{proof}

Для решения нормальных линейных систем методом матричной экспоненты мы будем находить собственные вектора.

Матрица $A$ в базисе из собственных векторов (если они соответствуют действительным собственным значениям) будет иметь диагональный вид. Произведение диагональной матрицы на диагональную -- диагональная. Тогда для случая без кратных корней:

\[ e^{tA} = E + t\cdot diag(\lambda_1, \dots, \lambda_n) + \dots + \frac{t^n}{n!}\cdot diag(\lambda_1^n, \dots , \lambda_n^n) + \dots\]

\[ e^{tA} = diag(e^{t\lambda_1}, \dots, e^{t\lambda_n}).\]

Если $\lambda$ -- корень кратности $l$, то матрица $A$ приводится к Жордановой клетке (диагональная матрица с единицами над главной диагональю).

\[ A = \lambda E + B \Rightarrow B = A - \lambda E. \]

\[ e^{tA} = e^{t(\lambda E + B)} = e^{t\lambda E}e^{tB},\ e^{t\lambda E} = diag(e^{t\lambda}, \dots, e^{t\lambda}), e^{tB} = E + tB + \dots + \frac{t^{l-1}}{(l-1)!}B^{l-1} + 0 \]

\begin{equation*}
	\text{тогда } e^{tA} = e^{\lambda t}
 	\begin{pmatrix}
            1\ \ t\ \ \ \ \  \dots\ \ \ \ \ \frac{t^{l-1}}{(l-1)!} \\
            0\ \ 1\ \ t\ \ \ \dots\ \ \ \ \  \frac{t^{l-2}}{(l-2)!} \\
            \dots \\
            0\ \ \ \ \ \ \ \ \ \dots\ \ \ \ \ \ \ \ \  1
    \end{pmatrix}
\end{equation*}

\textbf{Метод решения линейных неоднородных уравнений с постоянными коэффициентами} (матричный метод вариации постоянной)

\[ \frac{d\overrightarrow{x}}{dt} = A\overrightarrow{x} + \overrightarrow{f}(t),\ \ \text{ решение будем искать в виде } \ \overrightarrow{x}(t) = e^{tA}\overrightarrow{C}(t), \]

\[ \text{ тогда } Ae^{tA}\overrightarrow{C}(t) + e^{tA}\dot{\overrightarrow{C}}(t) = Ae^{tA}\overrightarrow{C} + \overrightarrow{f}(t),\]

\[ e^{tA}\dot{\overrightarrow{C}}(t) = \overrightarrow{f}(t)\ \Rightarrow \dot{\overrightarrow{C}}(t) = (e^{tA})^{-1}\overrightarrow{f}(t) = e^{-tA}\overrightarrow{f}(t). \]

\newpage

\section{Билет 4. Линейные дифференциальные уравнения и линейные системы дифференциальных уравнений с переменными коэффициентами}

\subsection{Теоремы существования и единственности решения задачи Коши для нормальной линейной системы уравнений и
для линейного уравнения $n$-го порядка в нормальном виде}

Рассматривается система вида 

\begin{equation}
	\frac{d\overrightarrow{x}}{dt} = A\overrightarrow{x} + \overrightarrow{q}(t),
	\label{Issue5_2}
\end{equation} 

где $A = ||a_j^i(t)||$, $i,j = \overrightarrow{1,n}$ -- матрица, $\overrightarrow{q}(t)$ -- заданная вектор-фенкция. Наряду с векторной записью также будем использовать координатную запись $\dot{x}^i = \sum\limits_{j = 1}^{n} a_j^i x^j + q^i(t),\ i = \overrightarrow{1,n}$.

$\textbf{Необходимым условием линейности}$ является факт того, что все $A_j^i$ и $q^i$ зависят только от $t$ и не зависят от $\overrightarrow{x}$.

Для $(\ref{Issue5_2})$ ставится задача Коши:

\[ \overrightarrow{x}(t_0) = \overrightarrow{x_0}.\]

\begin{theorem}

$\textbf{Основная теорема для линейных систем.}$ Пусть $a_j^i(t),\ i,j = \overrightarrow{1,n}$ и $\overrightarrow{q}(t)$ в $(\ref{Issue5_2})$ непрерывны на отрезке $[a;b]$. Тогда рпешение задачи Коши существует и единственно на всем отрезке $[a;b].$

\end{theorem}

$\textbf{Предварительные замечания:}$

Пусть вектор-функция $\overrightarrow{f}(x) \in B$ и $A$ -- линейный оператор, действующий из $B$ в $B$, т.е. $A(\overrightarrow{f} + \overrightarrow{g}) = A\overrightarrow{f} + A\overrightarrow{g}$.

$\textbf{Определим норму оператора:}$

\[ ||A|| =  \sup\limits_{\overrightarrow{\varphi} \in B,\ \overrightarrow{\varphi} \neq \overrightarrow{0}} \frac{||A(\overrightarrow{\varphi})||}{||\overrightarrow{\varphi}||}. \]

Тогда получаем неравенство: $||A|| \\leqslant ||A||||\overrightarrow{\varphi}||$.

Нормой для вектор-функции выберем $||\overrightarrow{x}(t)|| = \max\limits_{1 \leqslant i \leqslant n} (\max\limits_{t \in [a;b]} x^i(t)),$ а нормой для оператора $||A|| = = \max\limits_{1 \leqslant i \leqslant n} (\max\limits_{t \in [a;b]} \sum\limits_{j = 1}^{n} |a_j^i(t)|)$

\begin{proof}

Определим $\overrightarrow{g}(t) = \overrightarrow{x_0} + \int\limits_{t_0}^{t} \overrightarrow{q}(S)dS$ и построим итерационную процедуру.

Т.к $q^i(t) \in C_{[a;b]}\ \forall i = \overrightarrow{1,n} \Rightarrow \exists ||\overrightarrow{q}||_c = M_1.$ Тогда $||\overrightarrow{g}||_c = ||\overrightarrow{x_0} + \int\limits_{t_0}^{t}\overrightarrow{q}(S)dS|| \leqslant ||\overrightarrow{x_0}|| + ||\int\limits_{t_0}^{t}\overrightarrow{q}(S)dS|| \leqslant ||\overrightarrow{x_0}|| + M_1(b-a) = C.$

Рассмотрим интегральное уравнение $\overrightarrow{x} = \overrightarrow{g} + \int\limits_{t_0}^{t}A(s)\overrightarrow{x}(s)ds$.

Аналогично основной лемме доказывается, что последнее интегральное уравнение эквивалентно задаче $(\ref{Issue5_2})$.

Итерационная процедура: $\overrightarrow{x_0} = \overrightarrow{g}$; $\overrightarrow{x_k} = \overrightarrow{g} + \int\limits_{t_0}^{t} A(s)\overrightarrow{x_{k-1}}(s)ds $, $k = 0,1,\ \dots$

Оценим норму:

$$ ||\overrightarrow{x_1} - \overrightarrow{x_0}|| = || \int\limits_{t_0}^{t} A(s)\overrightarrow{g}(s)ds || \leqslant | \int\limits_{t_0}^{t} ||A(s)\overrightarrow{g}(s)||ds | \leqslant  | \int\limits_{t_0}^{t} ||A(s)||\cdot ||\overrightarrow{g}(s)||ds | \leqslant C_1 C |t - t_0|;$$ 

Таким образом $||\overrightarrow{x_1} -\overrightarrow{x_0}|| \leqslant C_1C|t-t_0|.$

Теперь докажем по индукции неравенство: $||\overrightarrow{x_k} - \overrightarrow{x_{k-1}}|| \leqslant \frac{CC_1^k}{k!}|t-t_0|^k.$

Базой индукции выступает полученное выше неравенство. Предположим, что верно для $n = k$, т.е.: $||\overrightarrow{x_k} - \overrightarrow{x_{k-1}}|| \leqslant \frac{CC_1^k}{k!}|t-t_0|^k.$

Докажем для 

\[n = k + 1:\  ||\overrightarrow{x_{k+1}} - \overrightarrow{x_{k}}|| = || \int\limits_{t_0}^{t} A(s)(\overrightarrow{x_k}(s) - \overrightarrow{x_{k-1}}(s))ds || \leqslant | \int\limits_{t_0}^{t} ||A(s)(\overrightarrow{x_k}(s) - \overrightarrow{x_{k-1}}(s))|| ds | \leqslant \]

\[ \leqslant | \int\limits_{t_0}^{t} ||A(s)||\cdot ||(\overrightarrow{x_k}(s) - \overrightarrow{x_{k-1}}(s))|| ds | \leqslant C |\int\limits_{t_0}^{t} \frac{C_1C^k|s-t_0|^k}{k!}ds | = \frac{C^{k+1}C_1|t-t_0|^{k+1}}{(k+1)!} \]

Т.к. $|t-t_0| \leqslant (b-a)$, то предыдущее неравенство можно усилить $||\overrightarrow{x_k} - \overrightarrow{x_{k-1}}|| \leqslant \frac{C_1C^k}{k!}(b-a)^k.$

Функциональная последовательно $\overrightarrow{x_k}$ сходиться равномерно, т.к. сходится равномерно ряд $\overrightarrow{x_0} + (\overrightarrow{x_1} - \overrightarrow{x_0}) +\ \dots\ +(\overrightarrow{x_k} - \overrightarrow{x_{k-1}}) +\ \dots$, который межорируется сходящимся рядом $||\overrightarrow{x_0}|| + ||(\overrightarrow{x_1} - \overrightarrow{x_0})|| +\ \dots\ + ||(\overrightarrow{x_k} - \overrightarrow{x_{k-1}})|| +\ \dots \leqslant ||\overrightarrow{x_0}|| + C_1\sum\limits_{k = 0}^{\infty}\frac{C^k|b-a|^k}{k!} = ||\overrightarrow{x_0}|| + C_1 e^{C(b-a)} < \infty \Rightarrow$ Существует (в сиду банаховости пр-ва) непрерывно дифф. $\overrightarrow{\varphi(t)}:$ $\exists \lim\limits_{n \rightarrow \infty} \overrightarrow{x_n} = \varphi(t).$

Рассмотрим $|| \int\limits_{t_0}^{t} A\overrightarrow{x_n}dS - \int\limits_{t_0}^{t} A\overrightarrow{\varphi} dS || = || \int\limits_{t_0}^{t} A(\overrightarrow{x_n} - \overrightarrow{\varphi})dS || \leqslant ||A||\cdot | \int\limits_{t_0}^{t} ||\overrightarrow{x_n} - \overrightarrow{\varphi}|| dS | $, где $||\overrightarrow{x_n} - \overrightarrow{\varphi} || \rightarrow_{n\rightarrow \infty} 0.$

Т.о. итерационная процедура сходится в силу существования пределов слева и справа.

Полученное решение эквивалентно решению задачи $(\ref{Issue5_2})$. В отличии от основной теоремы для нормальных систем ДУ: $\dot{\overrightarrow{x}} = \overrightarrow{f}(t, \overrightarrow{x})$, где существование было получено только на отрезке Пеано, для СЛДУ существование решения доказано для всего отрезка $[a;b]$ -- промежутка, где $a_j^i(t)$ и $\overrightarrow{q}(t)$ непрерывны. В нашем случае $\overrightarrow{f}$ соответствует $\overrightarrow{f} = A\overrightarrow{x} + \overrightarrow{q}$. Она непрерывна, т.к. полученное решение $\overrightarrow{x}(t)$ непрерывно. Условие непрерывности $\frac{\partial f}{\partial x_i}$ также выполнены, т.к. в нашем случае $\frac{\partial f}{\partial x_i} = a_{ij}(t)$ -- непр. на $[a;b]$. Отсюда следует единственность, т.к. два решения задачи $(\ref{Issue5_2})$, согласно основной теореме для нормальныэ систем, совпадает на промежутке, где они оба определены. В нашем случае это $[a;b]$.

Т.о. теорема не носит локальных характер.

\end{proof}