\documentclass[a4paper, 12pt]{article}

%%% Работа с русским языком
\usepackage{cmap}					% поиск в PDF
\usepackage{mathtext} 				% русские буквы в формулах
\usepackage[T2A]{fontenc}			% кодировка
\usepackage[utf8]{inputenc}			% кодировка исходного текста
\usepackage[russian]{babel}	% локализация и переносы

%%% Дополнительная работа с математикой
\usepackage{amsmath,amsfonts,amssymb,amsthm,mathtools} % AMS
\usepackage{icomma} % "Умная" запятая: $0,2$ --- число, $0, 2$ --- перечисление

%% Номера формул
%\mathtoolsset{showonlyrefs=true} % Показывать номера только у тех формул, на которые есть \eqref{} в тексте.
%\usepackage{leqno} % Немуреация формул слева

%% Шрифты
\usepackage{euscript}	 % Шрифт Евклид
\usepackage{mathrsfs} % Красивый матшрифт

%%% Свои команды
\DeclareMathOperator{\sgn}{\mathop{sgn}}

%% Поля
\usepackage[left=2cm,right=2cm,top=2cm,bottom=2cm,bindingoffset=0cm]{geometry}

%% Русские списки
\usepackage{enumitem}
\makeatletter
\AddEnumerateCounter{\asbuk}{\russian@alph}{щ}
\makeatother

%%% Работа с картинками
\usepackage{graphicx}  % Для вставки рисунков
\graphicspath{{images/}{images2/}}  % папки с картинками
\setlength\fboxsep{3pt} % Отступ рамки \fbox{} от рисунка
\setlength\fboxrule{1pt} % Толщина линий рамки \fbox{}
\usepackage{wrapfig} % Обтекание рисунков и таблиц текстом

%%% Работа с таблицами
\usepackage{array,tabularx,tabulary,booktabs} % Дополнительная работа с таблицами
\usepackage{longtable}  % Длинные таблицы
\usepackage{multirow} % Слияние строк в таблице

%% Красная строка
\setlength{\parindent}{2em}

%% Интервалы
\linespread{1}
\usepackage{multirow}

%% TikZ
\usepackage{tikz}
\usetikzlibrary{graphs,graphs.standard}

%% Верхний колонтитул
\usepackage{fancyhdr}
\pagestyle{fancy}

%% Перенос знаков в формулах (по Львовскому)
\newcommand*{\hm}[1]{#1\nobreak\discretionary{}
	{\hbox{$\mathsurround=0pt #1$}}{}}

%% дополнения
\usepackage{float} %Добавляет возможность работы с командой [H] которая улучшает расположение на странице
\usepackage{gensymb} %Красивые градусы
\usepackage{caption} % Пакет для подписей к рисункам, в частности, для работы caption*

% подключаем hyperref (для ссылок внутри  pdf)
\usepackage[unicode, pdftex]{hyperref}

%%% Теоремы
\theoremstyle{plain}                    % Это стиль по умолчанию, его можно не переопределять.
\renewcommand\qedsymbol{$\blacksquare$} % переопределение символа завершения доказательства

\newtheorem{theorem}{Теорема}[section] % Теорема (счетчик по секиям)
\newtheorem{proposition}{Утверждение}[section] % Утверждение (счетчик по секиям)
\newtheorem{definition}{Определение}[section] % Определение (счетчик по секиям)
\newtheorem{corollary}{Следствие}[theorem] % Следстиве (счетчик по теоремам)
\newtheorem{problem}{Задача}[section] % Задача (счетчик по секиям)
\newtheorem*{remark}{Примечание} % Примечание (можно переопределить, как Замечание)
\newtheorem{lemma}{Лемма}[section] % Лемма (счетчик по секиям)

\newtheorem{example}{Пример}[section] % Пример
\newtheorem{counterexample}{Контрпример}[section] % Контрпример
\newcommand{\defeq}{\stackrel{def}{=}} % по определению
\newcommand{\defarr}{\stackrel{def}{\Rightarrow}} % следует из определения
\DeclareMathOperator{\diverg}{div} % определение нормально выглядещей дивергенции

\begin{document}
\subsection{Фундаментальная система решений и общее решение нормальной линейной однородной
системы уравнений}

Рассмотрим систему вида 
\begin{equation}
    \label{eq7:SLDE}
    \dot{\vec{x}} = A \vec{x} + \vec{f},
\end{equation} 
где $A = || a^i_j||$, $i,\,j = \overline{1, n}$ - матрица системы, 
причём $a^i_j$ - числа; 
$ \vec{f}(t) = 
  \begin{Vmatrix}
    f^1(t) \\
    \cdots    \\
    f^n(t)
  \end{Vmatrix}$ - вектор-столбец неоднородной системы;
$\vec{x}(t) = 
\begin{Vmatrix}
  x^1(t) \\
  \cdots    \\
  x^n(t)
\end{Vmatrix}$ - вектор-столбец искомых функций.  


Наряду с вышеприведённой записью также будем рассматривать запись вида: 
$$\frac{dx^i}{dt} = \sum\limits^n_{j=1}a^i_j x^j(t) + f^i, ~i = \overline{1, n}$$

Основная идея решения систем дифференциальных уравнений вида \eqref{eq7:SLDE}, 
состоит в том, что матрица системы рассматривается как матрица линейного преобразования 
линейного пространства $\vec{\mathbb{R}}^n$ (пространство, присоединнёное к аффинному 
$\mathbb{R}^n$), заданная в исходном базисе. 

Пусть $S = \begin{Vmatrix} \sigma_j^i \end{Vmatrix}$, $i,\,j = \overline{1, n}$ - матрица перехода от исходного базиса $\begin{Vmatrix} \vec{e_1}, ..., \vec{e_n} \end{Vmatrix}$ к базису. 
Эти соотношения связаны выражением $\begin{Vmatrix} \vec{e_1}, ..., \vec{e_n} \end{Vmatrix} = \begin{Vmatrix} \vec{e'_1}, ..., \vec{e'_n} \end{Vmatrix} \cdot S $ 
или $\vec{e'_i} = \sum\limits_{k = 1}^n \sigma_i^k \vec{e_k}$, а координаты векторов в новом и старом базисе связаны формулой $\vec{x} = S \vec{x'}$ или $x^i = \sum\limits_{m = 1}^n \sigma_m^i {x'}^m$.

Матрица перехода $S$ обратима, поэтому $\exists S^{-1} = \begin{Vmatrix} \tau_j^i \end{Vmatrix}$, $i,\,j = \overline{1, n}$, причём $SS^{-1} = S^{-1}S = E$, 
т.е. $\sum \limits_{k = 1}^n \tau_k^i \sigma_j^k = \delta_j^i$. Тогда $\vec{x'} = S^{-1}\vec{x}$.
Преобразуем исходную систему, умножив её справа на $S^{-1}$.

\[ S^{-1} \frac{d\vec{x}}{dt} = \frac{d}{dt} (S^{-1}\vec{x}) = S^{-1}A\vec{x} + S^{-1}\vec{f}\]

Подставив $\vec x = S \vec{\bar{x}}$, получим $\frac{d\vec{\bar{x}}}{dt} = \bar{A} \vec{x} + \vec{\bar{f}}$, где $\vec{\bar{f}}(t) = S^{-1}\vec{f}(t)$, 
а $\bar{A} = S^{-1}AS$ является матрицей преобразования $A$ в новом базисе. Уравнение имеет \textbf{ковариантный вид}, поэтому задачи свелись к нахождению базиса, в котором система имела бы наиболее простой вид.

Пусть $A$ - матрица системы \eqref{eq7:SLDE} является матрицей линейного преобразования линейного пространства $\vec{\mathbb{R}}^n$, 
т.е. $\forall \vec{x} \in \vec{\mathbb{R}}^n \mapsto A\vec{x} = \vec{y} \in \vec{\mathbb{R}}^n$, тогда $A = \begin{Vmatrix} A\vec{e_1}, ..., A\vec{e_n} \end{Vmatrix}$, 
т.е столбцы матрицы $A$ являются компонентами образов базисных векторов.


\begin{definition}
    Подпространство $L \subset \vec{\mathbb{R}}^n$ называется \textbf{инвариантным} подпространством относительно преобразования $A$, если $\forall \vec{x} \in L \mapsto A \vec{x} \in L$.
\end{definition}

Пусть $\vec{e}_1, ..., \vec{e}_s, \vec{e}_{s+1}, ..., \vec{e}_n$ - базис в $\vec{\mathbb{R}}^n$, а $\vec{e}_1, ..., \vec{e}_s$ - базис в $L$. \\
Тогда $\forall i = \overline{1, s} \mapsto A\vec{e_i} = \sum\limits_{k=1}^s \gamma_i^k \vec{e_k}$ и матрица $A$ в этом базисе будет иметь вид:

\[ A = \begin{Vmatrix} A_1 & A_2 \\ O & A_3 \end{Vmatrix}, \text{ где } A_1 = \begin{Vmatrix} \gamma_1^1 & \cdots & \gamma_s^1 \\ \vdots & & \vdots \\ \gamma_1^s &\cdots & \gamma_s^s\ \end{Vmatrix}, ~O \text{ - нулевая матрица размером } (n - s) \times s. \]

Если $\vec{\mathbb{R}}^n = L^1 \oplus ... \oplus L^k$ и $L^i, ~i = \overline{1, k}$ - инвариантные подпространства, то в базисе, который является базисом-объединения всех базисотв инваритных подпространств, прямая сумма которых равна $\vec{\mathbb{R}}^n$, матрица будет иметь вид:
\[A = \begin{Vmatrix} A_1 & O & \cdots & O \\ O & A_2 & \cdots & O \\ \cdots & \cdots & \cdots & \cdots \\ O & O &\cdots & A_k \end{Vmatrix} \] 

$A_i, ~i = \overline{1, k}$ - квадратная матрица размерами $l_i < n$, которая является сужением матрицы преобразования $A$ на инвариантное подпространство $L_i$

В таком случае искомую вектор-функцию можно переписать в виде: 
\[ \vec{x}(t) = \begin{Vmatrix} x^1 \\ \cdots \\ x^{l_1} \\ \cdots \\ x^{l_1 + ... + l_{i-1} + 1} \\ x^{l_1 + ... + l_{i}} \\ \cdots \\ x^{l_1 + .. l_k + 1} \\ \cdots \\ x^n \end{Vmatrix}\]

\[\text{Обозначим через } X_i = \begin{Vmatrix} x^{l_1 + ... + l_{i-1} + 1} \\ \cdots \\ x^{l_1 + .. + l_{i-1} + l_{i}}\end{Vmatrix} \]

Тогда система \eqref{eq7:SLDE} распадается на $k$ систем, порядок которых $l_i < n$: \\
$\dot{\vec{X_i}} = A_i \vec{X_i} + \vec{f_i}(t), ~i =\overline{1, k}$

Для приведения матрицы линейного преобразования к клеточно-диагональному виду нужно найти собственные векторы линейного преобразования. 
Вектор $\vec{x} \neq 0$ называется собственнным вектором линейного преобразования, матрица которого равна $A$, если $A\vec{x} = \lambda \vec{x}$. 
Пусть $A = \begin{Vmatrix} a_j^i \end{Vmatrix}, ~i,\,j = \overline{1, n}$, а $\begin{Vmatrix} x^1 \\ \cdots \\ x^n \end{Vmatrix}$ - компоненты собственного вектора.
Тогда компоненты собственного вектора должны удовлетворять системе однородных линенейных уравнений вида $|| A - \lambda E|| \vec{x} = 0$. 
Чтобы эта система имела ненулевое решение необходимо, чтобы $\det || A - \lambda E|| = P_n(\lambda) = (-1)^n \lambda^n + (-1)^{n-1} \Tr A + ... + \det A = 0$. \\
$P_n(\lambda)$ - характерестический многочлен матрицы $A$. 

\subsubsection*{Случай простых корней характеристического многочлена}
Рассмотрим однорудную систему с постоянными коэффициентами \begin{equation}
  \dot{\vec{x}} = A \vec{x} \label{eq:ODN}
\end{equation}. Задача состоит в том, чтобы найти вектор функции $\vec{x}_1, \, \dots, \, \vec{x}_n$, которые будут образовывать ФСР нашей системы. 

\subsubsection*{Корни характеристического многочлена $\lambda_1, \dots, \lambda_n$ простые и действительные.}

Таким $\lambda_1, \dots, \lambda_n$ соответствуют собственные векторы $\vec{h}_1, \dots, \vec{h}_n$ ($A \vec{h}_i = \lambda_i \vec{h}_i$)
Можно показать, что собственные вектора, соответствующие разным собственным значениям линейно независимы, 
поэтому существует базис из собственнных векторов $\vec{h}_1, \dots, \vec{h}_n$, в котором матрица $A$ имеет вид: 
$\bar{A} = \begin{Vmatrix} \lambda_1 & & & \\  & \lambda_2 & \\ & & \ddots &  \\ & & & \lambda_n \end{Vmatrix}$ 
Тогда система \eqref{eq:ODN} будет иметь следующий вид: 
\[ \begin{cases}
    \cfrac{d \vec{x\,}^1}{dt} = \lambda_1 \vec{x \,}^1 \\
    \cdots \\
    \cfrac{d \vec{x \,}^n}{dt} = \lambda_n \vec{x \,}^n
\end{cases} \Longrightarrow \]

вектор-функции $\varphi_1 = \begin{Vmatrix*} 1 \\ 0 \\ \vdots \\ 0 \end{Vmatrix*} e^{\lambda_1 t}$, ..., 
$\varphi_n = \begin{Vmatrix*} 0 \\ 0 \\ \vdots \\ 1 \end{Vmatrix*} e^{\lambda_n t}$ 
образует ФСР этой системы, т.к. являются линейно независимыми решениями.
Матрица перехода в этом случае $S = \begin{Vmatrix*} \vec{h}_1,  ...,  \vec{h}_n \end{Vmatrix*}$. 
Тогда получим, что \begin{equation}
  \vec{x}_1 = \vec{h}_1 e^{\lambda_1 t}, ..., \vec{x}_n = \vec{h}_n e^{\lambda_n t} \label{eq7:FSR}
\end{equation} является ФСР \eqref{eq:ODN}, т.к. $\vec{x}_i, \,i= \overline{1, n}$ из \eqref{eq7:FSR} являются решениями \eqref{eq:ODN}, 
линейная независимость вектор-функций $\vec{x}_1, ..., \vec{x}_n$ следует из того, что вронскиан \eqref{eq7:FSR} при $t=0$ является $det S \neq 0$ (свойство 10 вронскиана).
Тогда любое решение \eqref{eq:ODN} представимо в виде
\begin{equation}
  \vec{x} = c_1 \vec{h}_1 e^{\lambda_1 t} + ... + c_n \vec{h}_n e^{\lambda_n t}
\end{equation}

Можно доказать, что $\vec{x}_1, ..., \vec{x}_n$ - ФСР иначе:
\begin{lemma}
  Система функций $e^{\lambda_1t}, ..., e^{\lambda_n t}$, где все $\lambda_i$ - разные, является линейно независимой.  
\end{lemma}
\begin{proof}
  Составим линейную комбинацию, равную нулю: $c_1 e^{\lambda_1 t} + ... + c_n e^{\lambda_n t} = 0$ - продифференцируем $(n-1)$ раз и 
  запишем получившуюся систему для поиска $c_1, ..., c_n$

  \[
  \begin{cases}
    c_1 e^{\lambda_1 t} + ... + c_n e^{\lambda_n t} = 0 \\
    \lambda_1 c_1 e^{\lambda_1 t} + ... + \lambda_n c_n e^{\lambda_n t} = 0 \\
    \cdots \\
    \lambda_1^{n-1} c_1 e^{\lambda_1 t} + ... + \lambda_n^{n-1} c_n e^{\lambda_n t} = 0
  \end{cases} 
  \]

Система является однородной, поэтому имеет тривиальное решение, но единственное ли оно? 

\[ \Delta = \begin{vmatrix*}
  e^{\lambda_1 t} & \cdots & e^{\lambda_n t} \\
  \lambda_1 e^{\lambda_1 t} & \cdots & \lambda_n e^{\lambda_n t} \\
  \vdots &  & \vdots \\ 
  \lambda_1^{n-1} e^{\lambda_1 t} & \cdots & \lambda_n^{n-1} e^{\lambda_n t} \\
\end{vmatrix*} = e^{\lambda_1 t + ... + \lambda_n t} \begin{vmatrix*}
  1 & \cdots & 1 \\
  \lambda_1  & \cdots & \lambda_n  \\
  \vdots &  & \vdots \\ 
  \lambda_1^{n-1}  & \cdots & \lambda_n^{n-1} \\
\end{vmatrix*} = e^{\lambda_1 t + ... + \lambda_n t} \prod \limits_{1 \leq j < i \geq n} (\lambda_i - \lambda_j) \neq 0
\]

Полученный определитель это определитель Вандермонда, который равен нулю только, если какая-то пара $\lambda_i, \lambda_j$ совпадёт. 
Значит определитель не равен нулю по условию $\Rightarrow$ система имеет только тривиальное решение по теореме Крамера $\Rightarrow$ система линейно независима.
\end{proof}

\begin{lemma}
Система $\vec{\varphi}_1 = \vec{h}_1 e^{\lambda_1 t}, ..., \vec{\varphi}_n = \vec{h}_n e^{\lambda_n t}$ является ФСР. 
\end{lemma}

\begin{proof}
  $\vec{\varphi}_i = \vec{h}_i e^{\lambda_i t}$ является решением по построению. Рассмотрим $W(t)$: $W(t) = \begin{vmatrix*} \vec{h}_1 e^{\lambda_1 t} \,...\, \vec{h}_n e^{\lambda_n t}\end{vmatrix*}$, 
  при $t = 0$: $W(0) = \begin{vmatrix*} \vec{h}_1 \, ... \, \vec{h}_n \end{vmatrix*} \neq 0$, т.к. собственные вектора линейно независимые. 
  Следовательно, по 10 свойству определителя Вронского система линейно независимая.
\end{proof}

Итак, общее решение системы \eqref{eq:ODN} записывается в виде: 

\begin{equation*}
  \boxed{\vec{x}^{\text{ об}}_0 = c_1 \vec{h}_1 e^{\lambda_1 t} + ... + c_n \vec{h}_n e^{\lambda_n t}}
\end{equation*}

\subsubsection*{Корни характеристического многочлена $\lambda_1, \dots, \lambda_n$ простые, но среди них есть комплексные.}
Пусть есть комплексные собственное число $\lambda_k = r_k + i \omega_k$ и ему соответствующий комплесный собственный вектор $\vec{h}_k + i \vec{d}_k $, 
где $\vec{h}_k$, $\vec{d}_k$ - действительные вектора. Так как характеристический многочлен это многочлен с действительными коэффициентами, 
то комплексгый корень идет вместе с комплексно ему сопряженным, т.е. $\bar{\lambda}_k = r_k - i \omega_k$ тоже является корнем характеристического многочлена. 

Взяв комплексное сопряжение над равенством $A (\vec{h}_k + i \vec{d}_k) = (r_k + i \omega_k)(\vec{h}_k + i \vec{d}_k)$:

\[ \overline{A (\vec{h}_k + i \vec{d}_k)} = A (\vec{h}_k - i \vec{d}_k) = \overline{(r_k + i \omega_k)(\vec{h}_k + i \vec{d}_k)} = (r_k - i \omega_k)(\vec{h}_k - i \vec{d}_k) \]
Т.е. $\vec{h}_k - i \vec{d}_k$ является собственным вектором для $\vec{\lambda_k} = r_k - i \omega_k$.

Аналогично случайно действительных простых корней система принимает вид: 

\[ \begin{cases}
  \cfrac{d \vec{x}_1}{dt} = \lambda_1 \vec{x}_1 \\ 
  \ldots \\
  \cfrac{d \vec{x}_k}{dt} = (r_k + i \omega_k)\vec{x}_k \\
  \cfrac{d \vec{x}_{k+1}}{dt} = (r_k - i \omega_k)\vec{x}_{k+1} \\
  \ldots \\
  \cfrac{d \vec{x}_n}{dt} = \lambda_n \vec{x}_n
\end{cases}\]

ФСР такой системы будет комплексной: $\begin{vmatrix*} 1 \\ 0 \\ 0 \\ \vdots \\ \vdots \\ 0 \\ 0 \end{vmatrix*} e^{\lambda_1 t}$; ...; 
$\begin{vmatrix*} 0 \\ \vdots \\ 0 \\ 1 \\ 0 \\ \vdots \\ 0 \end{vmatrix*} e^{r_k t} (cos \omega_k t + i sin \omega_k t)$; \\
$\begin{vmatrix*} 0 \\ \vdots \\ 0 \\ 0 \\ 1 \\ \vdots \\ 0 \end{vmatrix*} e^{r_k t} (cos \omega_k t - i sin \omega_k t)$; ...; 
$\begin{vmatrix*} 0 \\ 0 \\ 0 \\ \vdots \\ \vdots \\ 0 \\ 1 \end{vmatrix*} e^{\lambda_n t}$

Т.к. матрица перехода $S = \begin{Vmatrix*} \vec{h}_1, \ldots, \vec{h}_k + i \vec{d}_k, \vec{h}_k - i \vec{d}_k, \ldots, \vec{h}_n \end{Vmatrix*}$, 
то комплексная ФСР \eqref{eq:ODN} будет: $\vec{h}_1 e^{\lambda_1 t}$, ..., $(\vec{h}_k + i \vec{d}_k) e^{r_k t} (cos\omega_k t + i sin \omega_k t)$, 
$(\vec{h}_k - i \vec{d}_k) e^{r_k t} (cos\omega_k t - i sin \omega_k t)$, ..., $\vec{h}_n e^{\lambda_n t}$  

Рассмотрим систему функций, у которых первые $k-1$ функции являются функциями построенной выше системы. В качестве $k$-ой и $k+1$-ой функций возьмём:

\[
   \vec{q}_k = \frac{1}{2}((\vec{h}_k + i \vec{d}_k) e^{r_k t} (cos \omega_k t + i sin \omega_k t) + (\vec{h}_k - i \vec{d}_k) e^{r_k t} (cos \omega_k t - i sin \omega_k t)) = e^{r_k t} (\vec{h}_k cos \omega_k t - \vec{d}_k sin \omega_k t)
\]

\[
   \vec{q}_{k+1} = \frac{1}{2i}((\vec{h}_k + i \vec{d}_k) e^{r_k t} (cos \omega_k t + i sin \omega_k t) - (\vec{h}_k - i \vec{d}_k) e^{r_k t} (cos \omega_k t - i sin \omega_k t)) = e^{r_k t} (\vec{h}_k sin \omega_k t + \vec{d}_k cos \omega_k t)
\]

Остальные вектор-функции оставим прежними. Так построенная система будет линейно независимой, т.к. была получена линейными комбинациями линейно независимых вектор-функций. Каждая функция данной системы будет решением \eqref{eq:ODN} по построению и принципу суперпозиции $\Rightarrow$ полученная система является ФСР \eqref{eq:ODN} и содержит только действительные функции $\Rightarrow$ 

\begin{equation*}
  \boxed{\vec{x}^{\text{ об}}_0 = c_1 \vec{h}_1 e^{\lambda_1 t} + ...+ c_k e^{r_k t} (\vec{h}_k cos \omega_k t - \vec{d}_k sin \omega_k t) + c_{k+1} e^{r_k t} (\vec{h}_k sin \omega_k t + \vec{d}_k cos \omega_k t) + ... + c_n \vec{h}_n e^{\lambda_n t}}
\end{equation*}

\subsubsection*{Случай кратных корней характеристического многочлена}

\subsection{Линейная неоднородная система уравнений в случае, когда неоднородность представлена векторным квазимногочленом}
\end{document}