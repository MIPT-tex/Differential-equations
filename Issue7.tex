\documentclass[a4paper, 12pt]{article}

%%% Работа с русским языком
\usepackage{cmap}					% поиск в PDF
\usepackage{mathtext} 				% русские буквы в формулах
\usepackage[T2A]{fontenc}			% кодировка
\usepackage[utf8]{inputenc}			% кодировка исходного текста
\usepackage[russian]{babel}	% локализация и переносы

%%% Дополнительная работа с математикой
\usepackage{amsmath,amsfonts,amssymb,amsthm,mathtools} % AMS
\usepackage{icomma} % "Умная" запятая: $0,2$ --- число, $0, 2$ --- перечисление

%% Номера формул
%\mathtoolsset{showonlyrefs=true} % Показывать номера только у тех формул, на которые есть \eqref{} в тексте.
%\usepackage{leqno} % Немуреация формул слева

%% Шрифты
\usepackage{euscript}	 % Шрифт Евклид
\usepackage{mathrsfs} % Красивый матшрифт

%%% Свои команды
\DeclareMathOperator{\sgn}{\mathop{sgn}}

%% Поля
\usepackage[left=2cm,right=2cm,top=2cm,bottom=2cm,bindingoffset=0cm]{geometry}

%% Русские списки
\usepackage{enumitem}
\makeatletter
\AddEnumerateCounter{\asbuk}{\russian@alph}{щ}
\makeatother

%%% Работа с картинками
\usepackage{graphicx}  % Для вставки рисунков
\graphicspath{{images/}{images2/}}  % папки с картинками
\setlength\fboxsep{3pt} % Отступ рамки \fbox{} от рисунка
\setlength\fboxrule{1pt} % Толщина линий рамки \fbox{}
\usepackage{wrapfig} % Обтекание рисунков и таблиц текстом

%%% Работа с таблицами
\usepackage{array,tabularx,tabulary,booktabs} % Дополнительная работа с таблицами
\usepackage{longtable}  % Длинные таблицы
\usepackage{multirow} % Слияние строк в таблице

%% Красная строка
\setlength{\parindent}{2em}

%% Интервалы
\linespread{1}
\usepackage{multirow}

%% TikZ
\usepackage{tikz}
\usetikzlibrary{graphs,graphs.standard}

%% Верхний колонтитул
\usepackage{fancyhdr}
\pagestyle{fancy}

%% Перенос знаков в формулах (по Львовскому)
\newcommand*{\hm}[1]{#1\nobreak\discretionary{}
	{\hbox{$\mathsurround=0pt #1$}}{}}

%% дополнения
\usepackage{float} %Добавляет возможность работы с командой [H] которая улучшает расположение на странице
\usepackage{gensymb} %Красивые градусы
\usepackage{caption} % Пакет для подписей к рисункам, в частности, для работы caption*

% подключаем hyperref (для ссылок внутри  pdf)
\usepackage[unicode, pdftex]{hyperref}

%%% Теоремы
\theoremstyle{plain}                    % Это стиль по умолчанию, его можно не переопределять.
\renewcommand\qedsymbol{$\blacksquare$} % переопределение символа завершения доказательства

\newtheorem{theorem}{Теорема}[section] % Теорема (счетчик по секиям)
\newtheorem{proposition}{Утверждение}[section] % Утверждение (счетчик по секиям)
\newtheorem{definition}{Определение}[section] % Определение (счетчик по секиям)
\newtheorem{corollary}{Следствие}[theorem] % Следстиве (счетчик по теоремам)
\newtheorem{problem}{Задача}[section] % Задача (счетчик по секиям)
\newtheorem*{remark}{Примечание} % Примечание (можно переопределить, как Замечание)
\newtheorem{lemma}{Лемма}[section] % Лемма (счетчик по секиям)

\newtheorem{example}{Пример}[section] % Пример
\newtheorem{counterexample}{Контрпример}[section] % Контрпример
\newcommand{\defeq}{\stackrel{def}{=}} % по определению
\newcommand{\defarr}{\stackrel{def}{\Rightarrow}} % следует из определения
\DeclareMathOperator{\diverg}{div} % определение нормально выглядещей дивергенции

\begin{document}
\subsection{Фундаментальная система решений и общее решение нормальной линейной однородной
системы уравнений}

Рассмотрим систему вида 
\begin{equation}
    \label{eq7:SLDE}
    \dot{\vec{x}} = A \vec{x} + \vec{f},
\end{equation} 
где $A = || a^i_j||$, $i,\,j = \overline{1, n}$ - матрица системы, 
причём $a^i_j$ - числа; 
$ \vec{f}(t) = 
  \begin{Vmatrix}
    f^1(t) \\
    \cdots    \\
    f^n(t)
  \end{Vmatrix}$ - вектор-столбец неоднородной системы;
$\vec{x}(t) = 
\begin{Vmatrix}
  x^1(t) \\
  \cdots    \\
  x^n(t)
\end{Vmatrix}$ - вектор-столбец искомых функций.  


Наряду с вышеприведённой записью также будем рассматривать запись вида: 
$$\frac{dx^i}{dt} = \sum\limits^n_{j=1}a^i_j x^j(t) + f^i, ~i = \overline{1, n}$$

Основная идея решения систем дифференциальных уравнений вида \eqref{eq7:SLDE}, 
состоит в том, что матрица системы рассматривается как матрица линейного преобразования 
линейного пространства $\vec{\mathbb{R}}^n$ (пространство, присоединнёное к аффинному 
$\mathbb{R}^n$), заданная в исходном базисе. 

Пусть $S = \begin{Vmatrix} \sigma_j^i \end{Vmatrix}$, $i,\,j = \overline{1, n}$ - матрица перехода от исходного базиса $\begin{Vmatrix} \vec{e_1}, ..., \vec{e_n} \end{Vmatrix}$ к базису. 
Эти соотношения связаны выражением $\begin{Vmatrix} \vec{e_1}, ..., \vec{e_n} \end{Vmatrix} = \begin{Vmatrix} \vec{e'_1}, ..., \vec{e'_n} \end{Vmatrix} \cdot S $ 
или $\vec{e'_i} = \sum\limits_{k = 1}^n \sigma_i^k \vec{e_k}$, а координаты векторов в новом и старом базисе связаны формулой $\vec{x} = S \vec{x'}$ или $x^i = \sum\limits_{m = 1}^n \sigma_m^i {x'}^m$.

Матрица перехода $S$ обратима, поэтому $\exists S^{-1} = \begin{Vmatrix} \tau_j^i \end{Vmatrix}$, $i,\,j = \overline{1, n}$, причём $SS^{-1} = S^{-1}S = E$, 
т.е. $\sum \limits_{k = 1}^n \tau_k^i \sigma_j^k = \delta_j^i$. Тогда $\vec{x'} = S^{-1}\vec{x}$.
Преобразуем исходную систему, умножив её справа на $S^{-1}$.

\[ S^{-1} \frac{d\vec{x}}{dt} = \frac{d}{dt} (S^{-1}\vec{x}) = S^{-1}A\vec{x} + S^{-1}\vec{f}\]

Подставив $\vec x = S \vec{\bar{x}}$, получим $\frac{d\vec{\bar{x}}}{dt} = \bar{A} \vec{x} + \vec{\bar{f}}$, где $\vec{\bar{f}}(t) = S^{-1}\vec{f}(t)$, 
а $\bar{A} = S^{-1}AS$ является матрицей преобразования $A$ в новом базисе. Уравнение имеет \textbf{ковариантный вид}, поэтому задачи свелись к нахождению базиса, в котором система имела бы наиболее простой вид.

Пусть $A$ - матрица системы \eqref{eq7:SLDE} является матрицей линейного преобразования линейного пространства $\vec{\mathbb{R}}^n$, 
т.е. $\forall \vec{x} \in \vec{\mathbb{R}}^n \mapsto A\vec{x} = \vec{y} \in \vec{\mathbb{R}}^n$, тогда $A = \begin{Vmatrix} A\vec{e_1}, ..., A\vec{e_n} \end{Vmatrix}$, 
т.е столбцы матрицы $A$ являются компонентами образов базисных векторов.


\begin{definition}
    Подпространство $L \subset \vec{\mathbb{R}}^n$ называется \textbf{инвариантным} подпространством относительно преобразования $A$, если $\forall \vec{x} \in L \mapsto A \vec{x} \in L$.
\end{definition}

Пусть $\vec{e_1}, ..., \vec{e_s}, \vec{e_{s+1}}, ..., \vec{e_n}$ - базис в $\vec{\mathbb{R}}^n$, а $\vec{e_1}, ..., \vec{e_s}$ - базис в $L$. \\
Тогда $\forall i = \overline{1, s} \mapsto A\vec{e_i} = \sum\limits_{k=1}^s \gamma_i^k \vec{e_k}$ и матрица $A$ в этом базисе будет иметь вид:

\[ A = \begin{Vmatrix} A_1 & A_2 \\ O & A_3 \end{Vmatrix}, \text{ где } A_1 = \begin{Vmatrix} \gamma_1^1 & \cdots & \gamma_s^1 \\ \vdots & & \vdots \\ \gamma_1^s &\cdots & \gamma_s^s\ \end{Vmatrix}, ~O \text{ - нулевая матрица размером } (n - s) \times s. \]

Если $\vec{\mathbb{R}}^n = L^1 \oplus ... \oplus L^k$ и $L^i, ~i = \overline{1, k}$ - инвариантные подпространства, то в базисе, который является базисом-объединения всех базисотв инваритных подпространств, прямая сумма которых равна $\vec{\mathbb{R}}^n$, матрица будет иметь вид:
\[A = \begin{Vmatrix} A_1 & O & \cdots & O \\ O & A_2 & \cdots & O \\ \cdots & \cdots & \cdots & \cdots \\ O & O &\cdots & A_k \end{Vmatrix} \] 

$A_i, ~i = \overline{1, k}$ - квадратная матрица размерами $l_i < n$, которая является сужением матрицы преобразования $A$ на инвариантное подпространство $L_i$

В таком случае искомую вектор-функцию можно переписать в виде: 
\[ \vec{x}(t) = \begin{Vmatrix} x^1 \\ \cdots \\ x^{l_1} \\ \cdots \\ x^{l_1 + ... + l_{i-1} + 1} \\ x^{l_1 + ... + l_{i}} \\ \cdots \\ x^{l_1 + .. l_k + 1} \\ \cdots \\ x^n \end{Vmatrix}\]

\[\text{Обозначим через } X_i = \begin{Vmatrix} x^{l_1 + ... + l_{i-1} + 1} \\ \cdots \\ x^{l_1 + .. + l_{i-1} + l_{i}}\end{Vmatrix} \]

Тогда система \eqref{eq7:SLDE} распадается на $k$ систем, порядок которых $l_i < n$: \\
$\dot{\vec{X_i}} = A_i \vec{X_i} + \vec{f_i}(t), ~i =\overline{1, k}$

Для приведения матрицы линейного преобразования к клеточно-диагональному виду нужно найти собственные векторы линейного преобразования. 
Вектор $\vec{x} \neq 0$ называется собственнным вектором линейного преобразования, матрица которого равна $A$, если $A\vec{x} = \lambda \vec{x}$. 
Пусть $A = \begin{Vmatrix} a_j^i \end{Vmatrix}, ~i,\,j = \overline{1, n}$, а $\begin{Vmatrix} x^1 \\ \cdots \\ x^n \end{Vmatrix}$ - компоненты собственного вектора.
Тогда компоненты собственного вектора должны удовлетворять системе однородных линенейных уравнений вида $|| A - \lambda E|| \vec{x} = 0$. 
Чтобы эта система имела ненулевое решение необходимо, чтобы $\det || A - \lambda E|| = P_n(\lambda) = (-1)^n \lambda^n + (-1)^{n-1} \Tr A + ... + \det A = 0$. \\
$P_n(\lambda)$ - характерестический многочлен матрицы $A$. 
\subsection{Линейная неоднородная система уравнений в случае, когда неоднородность представлена векторным квазимногочленом}
\end{document}