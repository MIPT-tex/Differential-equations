
\section{Билет 5. Автономные системы дифференциальных уравнений}

\subsection{Основные определения}
Система ДУ вида: 
\begin{equation}\label{avt_sys}
	\frac{dx^i}{dt} = f^i(x^1, ..., x^n); \quad \frac{d\vec{x}}{dt} = \vec{f}(\vec{x}); \quad 	\dot{x}^i = f^i(\vec{x}) \quad \quad i = \overline{1, n}
\end{equation}
называется автономной системой ДУ, если $ \vec{f} = \{f_i(x^1, ..., x^n)\} $,  $\ i = \overline{1, n}$ не зависит явно от аргумента $ t $; $ x^j = x^j(t), \ j = \overline{1, n} $ являются интегральными кривыми \eqref{avt_sys}. \\ $\vec{x}(t) = \{ x^j(t) \} \in \mathbb{R}^{n+1} = t \times \mathbb{R}^n$.
\begin{definition}
	Пусть $ \vec{x}(t) $ является решением \eqref{avt_sys}. Кривая $ \gamma $ в $ \mathbb{R}^n $ называется фазовой траекторией \eqref{avt_sys}. Само $ \mathbb{R}^n $ называется фазовым пространством \eqref{avt_sys}.
	\begin{equation}\label{gamma_sys}
		\gamma = \left\{
			\begin{aligned}
				x^1 &= x^1(t) \\
				x^2 &=x^2(t) \\
				.... \\
				x^n &= x^n(t) \\
			\end{aligned}
		\right.
	\end{equation}
	Будем предполагать, что $ \vec{f} = \{ f^i(x^1, ..., x^n) \} \in \mathscr{D} \subset \mathbb{R}^n, i = \overline{1, n} $ непрерывно дифференцируемые функции по всей совокупности переменных.
\end{definition}

\begin{theorem}
	Если $ \varphi(t) $ -- решение \eqref{avt_sys}, то $ \varphi(t + \tau) \ \forall \tau = const \in \mathbb{R}$ тоже решение \eqref{avt_sys}.
\end{theorem}

\begin{proof}
	\ \\
	Пусть $ u = t + \tau: \dfrac{d(\varphi(t + \tau))}{dt} = | t + \tau = u | = \dfrac{d \varphi(u)}{du} \dfrac{du}{dt} = \dfrac{d \varphi(u)}{du} = f(\varphi(u)) = f(\varphi(t + \tau))$, т.е. $\varphi(t + \tau)$ -- решение.
\end{proof}

\begin{corollary}
	Пусть $ \vec{\varphi}(t_0, \vec{x}_0)$ -- решение \eqref{avt_sys}, такое что $ \vec{\varphi}(t, t_0, \vec{x}_0) = \vec{x}_0 $. В силу доказанной теоремы $ \vec{\varphi}(t + \tau, t_0 + \tau, \vec{x}_0) $ тоже решение \eqref{avt_sys}. (Формально заменяем $ t + \tau $ на $ u $, $ t_0 + \tau $ на $ u_0$),  причём $ \vec{\varphi}(t_0 + \tau, t_0 + \tau, \vec{x}_0) = \vec{x}_0 $. Тогда, если $ \vec{f}(x^1,..., x^n) $ является непрерывной функцией n переменных вместе с $ \dfrac{\partial \vec{f}}{\partial x_i} $, то показанные решения совпадают по основной теореме. \\
	$ \vec{\varphi}(t + \tau, t_0 + \tau, \vec{x}_0) \equiv \vec{\varphi}(t, t_0, \vec{x}_0)$. Положим, в силу произвольности $ \tau $, $ \tau = - t_0 \Rightarrow$ \\ $\Rightarrow \vec{\varphi}(t, t_0, \vec{x}_0) = \vec{\varphi}(t - t_0, 0, \vec{x}_0) = \vec{\varphi}(t - t_0, \vec{x}_0) $ \\
	Т.о. положение движущейся по фазовой траектории точки определяется начальным положением $ \vec{x}_0 $ в момент времени $ t_0 $ и длительностью $ t - t_0 $, отсчитываемого от начального момента времени $ t_0 $, но не самим этим моментом. (Т.е. начальный момент не существенен и можно положить его равным нулю).
\end{corollary}

\begin{theorem}
	Фазовые траектории либо не имеют общих точек, либо совпадают
\end{theorem}

\begin{proof}
	\ \\
	Пусть $ \varphi (t)$ и $\psi(t) $ -- решения \eqref{avt_sys}, причём $ x_0 = \varphi(t_1) = \psi(t_2) $ Рассмотрим $ \chi (t) = \psi (t + (t_2 - t_1)) $, согласно предыдущей теореме $ \chi(t) $ тоже явл. реш. \eqref{avt_sys}, причём $ \chi (t_1) \overset{\text{по постр.}}{=\joinrel=\joinrel=} x_0 = \psi (t_2) =\varphi (t_1) \Rightarrow$ По основной теореме $ \varphi (t) \equiv \chi(t) \defeq \psi(t + (t_2 - t_1)) \Rightarrow $ траектории $ \varphi (t) $ и $ \psi(t) $ совпали.
\end{proof}
\noindent Согласно доказанному можно считать, что фазовое пространство \eqref{avt_sys} "склеено" из фазовых траекторий.

\subsection{Типы фазовых траекторий}

\begin{definition}
	Точка $\vec{a} \in \vec{\mathbb{R}}^n$ называется положением равновесия \eqref{avt_sys}, если \\ $ \vec{f}(\vec{a}) = 0 \ \ (f^i(a^1, ..., a^n) = 0, \ i = \overline{1, n} )$
\end{definition}

\begin{proposition}
	Если $ \vec{a} \in \vec{\mathbb{R}}^n $ -- положение равновесия \eqref{avt_sys}, то $ \vec{x}(t) = \vec{a}, -\infty < t < + \infty$ является решением \eqref{avt_sys}
\end{proposition}

\begin{proof}
	\ \\
	$\vec{x}(t) \equiv \vec{a} \overset{\eqref{avt_sys}}{\Rightarrow} 0 = \dfrac{d\vec{x}}{dt} = \dfrac{d \vec{a}}{dt} = f(\vec{a}) = 0 \Rightarrow$ удовлетворяет \eqref{avt_sys}
\end{proof}
\noindent Т.о. точка равновесия $ \vec{a} \in \vec{\mathbb{R}}^n$ является фазовой траекторией \eqref{avt_sys}

\begin{corollary}
	Решение \eqref{avt_sys} не может прийти в положение равновесия за конечное время.
\end{corollary}

\begin{proof}
	\ \\
	Пусть это не так и фазовая траектория пришла в положение равновесия за конечное время. Т.о., т.к. положение равновесия тоже является фазовой траекторией, то они пересекаются, что невозможно $\Rightarrow$ противоречие 
\end{proof}

\begin{theorem}
	Фазовые траектории принадлежат одному из трёх типов:
	\begin{enumerate}
		\item Точка (равновесия)
		\item Фазовая траектория, отличная от точки, есть гладкая кривая
		\item Замкнутая кривая(цикл) -- периодическая
	\end{enumerate}
\end{theorem}
