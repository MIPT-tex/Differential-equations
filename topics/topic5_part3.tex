\subsection{Устойчивость по Ляпунову. Асимптотическая устойчивость}
Рассматривается общая система дифференциальных уравнений
\begin{equation}
	\frac{d\vec{x}}{dt} = \vec{f}(t, x^1, \dots, x^n)
\end{equation}
Пусть $\vec{x} = \vec{\varphi}(t, \vec{x}_0)$ -- решение этой системы, такое что $\vec{\varphi}(t_0, \vec{x}_0) = \vec{x}_0$. А $\psi(t, \vec{\tilde{x}}_0)$ -- произвольное решение, такое что $\psi(t_0, \vec{\tilde{x}}_0) = \vec{\tilde{x}}_0$.
\begin{definition}
	Решение $\vec{x} = \vec{\varphi}(t, \vec{x}_0)$ называется \textbf{устойчивым по Ляпунову}, если
	\[
	\forall \varepsilon > 0 \; \exists \delta > 0:\forall \vec{\tilde{x}}_0: |\vec{\tilde{x}}_0 - \vec{x}_0|<\delta \to |\vec{\psi}(t, \vec{\tilde{x}}_0) -  \vec{\varphi}(t_0, \vec{x}_0)|<\varepsilon\text{ } \forall t\in [t_0, +\infty)
	\]
\end{definition}

\begin{definition}
	Решение $\vec{x} = \vec{\varphi}(t, \vec{x}_0)$ называется \textbf{асимптотически устойчивым}, если оно устойчиво по Ляпунову, а так же
	\[
	\forall \varepsilon > 0 \; \exists \delta > 0:\forall \vec{x}_0: |\vec{\tilde{x}}_0 - \vec{x}_0|<\delta \to \lim_{t\to+\infty}|\vec{\psi}(t, \vec{\tilde{x}}_0) -  \vec{\varphi}(t_0, \vec{x}_0)|=0
	\]
\end{definition}

\subsection{Автономные линейные системы}

Пусть в конечномерном линейном пространстве $B$ линейный оператор задается матрицей $A=||a_{ij}(t)||$. Если $a_{ij}$ ограничены, тогда норма матрицы 
\[
||A|| = \max_{i,j =\overline{1,n}}|\sup_{t\in I(t)}(a_{ij}(t))|
\]
Можно записать следующее неравенство:
\[
||A\vec{x}||\leqslant ||A||\cdot||\vec{x}||
\]

Теперь рассмотрим систему однородных уравнений, где $A$ постоянна
\begin{equation}
	\frac{d\vec{x}}{dt} = A\vec{x}
	\label{equ:issue12sys}
\end{equation}
Тогда $\vec{x} = 0$ -- решение.

\begin{lemma}
	Если однородная линейная система имеет неограниченное решение, то нулевое решение неустойчиво.
\end{lemma}

\begin{proof}
	Будем рассматривать систему (\ref{equ:issue12sys}). Пусть решение $\vec{\varphi}(t, \vec{x}_0)$ неограниченно. То есть 
	\[
	\forall M>0 \; \exists t^*:|\vec{\varphi}(t^*, \vec{x}_0)|>M
	\]
	
	Обратим определение устойчивости нулевого приближения 
	\[
	\exists \varepsilon_0>0: \forall \delta>0 \; \exists \vec{x}_0: |\vec{x}_0|<\delta, \exists t^*\in[t_0, +\infty): |\vec{\varphi}(t^*, \vec{x}_0)|>\varepsilon
	\]
	Воспользуемся неограниченностью решения
	\[
	\forall C>0 \to \vec{\psi}(t, \vec{x}_0) = C\cdot\vec{\varphi}(t, \vec{x}_0) \text{ -- неограниченно} 
	\]
	Теперь для произвольного $\delta >0 $ возьмем $C = \dfrac{\delta}{2|\vec{x}_0|}$
	\[
	|\vec{\psi}(t_0, \vec{x}_0)| = C\cdot|\vec{\varphi}(t_0, \vec{x}_0)| = \frac{\delta |\vec{x}_0|}{2|\vec{x}_0|} = \frac{\delta}{2}
	\]
	Таким образом
	\[
	\exists \varepsilon_0, \exists t: 	|\vec{\psi}(t, \vec{x}_0)| > \varepsilon_0
	\]
\end{proof}

\begin{theorem}
	Пусть $\lambda_1, \dots, \lambda_k$ -- собственные числа матрицы $A$ кратности $l_1, \dots, l_k$ соответственно. Тогда
	\begin{enumerate}
		\item Если $Re(\lambda_i)<0, i=\overline{1, k}$, то нулевое решение асимптотически устойчиво.
		\item Пусть $\exists j: \forall i\neq j :Re(\lambda_i) < 0, \; Re(\lambda_j) = 0$. И существует базис из собственных векторов $\vec{e}_{1}, \dots, \vec{e}_{l_j}$. Тогда нулевое решение устойчиво по Ляпунову.
		\item Если $\exists j: Re(\lambda_j) > 0$, или $Re(\lambda_j) = 0$, но собственные вектора не образуют базис, тогда нулевое решение не устойчиво
	\end{enumerate}
\end{theorem}

\begin{proof}[Доказательство (по Романко)]
	Рассмотрим случаи по порядку:
	\begin{enumerate}
		\item Если $Re(\lambda_i)<0 \; \forall i$ то $\exists \mu >0 : Re(\lambda_i)< -2\mu < 0$. Решение задачи Коши имеет вид 
		\[
			\vec{x}(t, \vec{x}_0) = \vec{x}_0\cdot e^{tA}
		\]
		Покажем, что $\exists M>0: \|e^{tA}\| \leqslant Me^{-\mu t}, \forall t>0$.
		
		Каждый элемент $a_{ij}(t)$ матричной экспоненты является конечной суммой квазимногочленов
		\[
			a_{ij}(t) = \sum_{k=1}^m P_k^{ij}(t)e^{\lambda_kt},\;  i,j=\overline{1,n}
		\]
		где $P_k^{ij}(t)$ - многочлены. Всегда найдется такое число $c_{ij}>0$, что для всех $k$
		\[
			|P_k^{ij}(t)e^{\lambda_kt}|\leqslant c_{ij}e^{-\mu t} \; \forall t
		\]
		Тогда
		\begin{multline*}
		|\vec{x}(t, \vec{x}_0)|\leqslant \|e^{tA}\|\cdot|\vec{x}_0| = |\vec{x}_0|\cdot\sqrt{\sum_{i,j=1}^n|a_{ij}(t)|^2}\leqslant\\\leqslant|\vec{x}_0|\cdot e^{-\mu t}\cdot m \sqrt{\sum_{i,j=1}^n|c_{ij}|^2} = Me^{-\mu t}\cdot |x_0|
		\end{multline*}
		Отсюда видно, что $|\vec{x}(t, \vec{x}_0)| \to 0$ при $t \to +\infty$
	\item 
		Решение задачи Коши имеет вид 
		\[
		\vec{x}(t, \vec{x}_0) = \vec{x}_0\cdot e^{tA}
		\]
		где элементы $a_{ij}(t)$ матричной экспоненты имеют вид
		\[
			a_{ij}(t) = \sum_{Re(\lambda) < 0}P_\lambda^{ij}(t)e^{\lambda t} + \sum_{Re(\lambda)=0}c_\lambda e^{\lambda t}
		\]
		Далее, аналогично предыдущему случаю получаем
		\[
			|\vec{x}(t, \vec{x}_0)| \leqslant \|e^{tA}\|\cdot|\vec{x}_0| \leqslant M\cdot|\vec{x}_0| 
		\]
		Откуда следует устойчивость по Ляпунову.
	\item
		Пусть $\exists \lambda = \mu + i \nu$ с $Re(\lambda) = \mu >0$. Тогда решения принимают вид
		\[
			\vec{x}(t, \vec{x}_0) = e^{\mu t}(\vec{h}_1\cos{\nu t}+ \vec{h}_2\sin{\nu t}), \; \vec{h}_1 = \vec{x}_0,
		\]
		где $\vec{h} = \vec{h}_1 + i\vec{h}_2$ - собственный вектор для $\lambda$ при $t = t_k$ и $\sin{\nu t_k} = 0$ получаем, что
		\[
			|\vec{x}(t, \vec{x}_0)| = e^{\mu t_k}|\vec{x}_0| \to +\infty,\; t_k\to+\infty.
		\]
		
		Если же $Re(\lambda) = 0$, то при условиях теоремы существует решение вида
		\[
			\vec{x}(t, \vec{x}_0) = e^{\lambda t}\left( \vec{h}_1 + t\vec{h}_2 + \cdots + \frac{t^{k-1}}{(k-1)!}\vec{h}_k\right), \vec{h}_1=x_0
		\]
		где $\vec{h}_1, \dots,\vec{h}_k$ - Жорданова цепочка для $\lambda$, причем $k>2$. 
		
		Отсюда ясно, что $|\vec{x}(t, \vec{x}_0)| \to \infty, t\to +\infty$
		
	\end{enumerate}
\end{proof}
