\subsection{Формула Лиувилля-Остроградского для нормальной линейной однородной системы уравнений и для линейного однородного уравнения $n$-го порядка.}

Следующее свойство вронскиана рассмотрим в виде теоремы. Для начала докажем вспомогательное утверждение.

\begin{lemma}

[Формула Эйлера дифференцирования определителя]\\
Детерминант матрицы представим в виде: 
$\Delta = 
\begin{vmatrix}
  a_1^1 & ... & a_n^1 \\
  a_1^i & ... & a_n^i \\
  ...   & ... & ...   \\
  a_1^n & ... & a_n^n \\
\end{vmatrix} = \sum\limits_{k = 1}^n{(-1)^{k + i} \cdot {a_k}^i {M_i}^k}
$
Тогда для 
\[\dot{\Delta}(t) = \sum\limits_{i = 1}^n{\sum\limits_{j = 1}^n{(-1)^{i + j} \cdot \dot{a_j^i}}M_i^j}\]

\end{lemma}

\begin{theorem}

[Формула Лиувилля-Остроградского]\\
Пусть $W(x) - $ вронскиан решений $\vec{\varphi_1}(t), ..., \vec{\varphi_n}(t)$ однородной системы $\dot{\vec{x}} = A\vec{x}$. Тогда имеет место формула:

\[\dot{W(t)} = W(t) \cdot trA\]
где $trA = \sum\limits_{k = 1}^n{a_{kk}(t)}$

\end{theorem}

\begin{proof}

Зафиксируем среди системы решений функцию $\vec{\varphi_j}=
\begin{pmatrix}
  \varphi_j^1 \\
  \varphi_j^2 \\
  ...   \\
  \varphi_j^n \\
\end{pmatrix}
$.
Рассмотрим i - ую компоненту $\varphi_j^i$ решения $\vec{\varphi_j}$. Поскольку $\vec{\varphi_j}$ решение, то $\frac{d\vec{\varphi_j}}{dt} = A\vec{\varphi_j} \Rightarrow$

\[\frac{d\varphi_j^i}{dt} = \dot{\varphi_j^i} = \sum\limits_{k = 1}^n{a_k^i \varphi_j^k}\]

Рассмотрим вранскиан $W(t)$, продифференцируем его по $t$

\[\dot{W}(t) = \sum\limits_{i = 1}^n{\sum\limits_{j = 1}^n{(-1)^{i + j} \cdot \dot{\varphi_j^i} M_j^i}} = \sum\limits_{i = 1}^n{\sum\limits_{j = 1}^n{\sum\limits_{k = 1}^n{(-1)^{i + j} \cdot a_k^i \varphi_j^k M_j^i}}}\]

Переставим суммы местами

\[\dot{W}(t) = \sum\limits_{k = 1}^n{\sum\limits_{i = 1}^n{a_k^i}{\sum\limits_{j = 1}^n{(-1)^{i + j} \varphi_j^k M_j^i}}} = \sum\limits_{k = 1}^n{\sum\limits_{i = 1}^n{a_k^i}\delta_i^k W(t)} = W(t)\sum\limits_{k = 1}^n{\sum\limits_{i = 1}^n{a_k^i}\delta_i^k} = W(t)\sum\limits_{k = 1}^n{a_k^k}\]

\[\dot{W}(t) = W(t) \cdot trA\]

\end{proof}

Также можно решить это уравнение и переписать в виде

\[W(t) = W(t_0)\exp{\left(\int\limits_{t_0}^{t}trA(u)du\right)}\]

\subsection{Метод вариации постоянных для линейной неоднородной системы уравнений и для линейного неоднородного уравнения n-го порядка.}

\begin{equation}\label{eq10_1}
\text{Рассмотрим} ~ y^{(n)} + a_1(x)y^{(n-1)} + ... + a_n(x)y = f(x).\\
\end{equation}

$\varphi_1(x),~...,~\varphi_n(x)$ -- Ф.С.Р. однородного уравнения $y^{(n)} + a_1(x)y^{(n-1)} + ... + a_n(x)y = 0$. Это означает, что 

\begin{equation}\label{eq10_2}
\forall k = \overline{1,n} \hookrightarrow \varphi_k^{(n)} + a_1(x)\varphi_k^{(n-1)} + ... + a_n(x)\varphi_k \equiv 0
\end{equation}

Перепишем уравнение $(\ref{eq10_1})$ в эквивалентном виде. Для этого сделаем следующие замены: $y = v_1, ~y^{(1)} = v_2,~..., ~y^{(n-1)} = v_n$. Тогда получим:

\begin{equation}\label{eq10_3}
 \begin{cases}
   \frac{dv_1}{dx} = v_2, 
   \\
   \frac{dv_2}{dx} = v_3,
   \\
   ...,
   \\
   \frac{dv_n}{dx} = f(x) - a_1(x)v_n - ... - a_n(x)v_1.
 \end{cases}
\end{equation}

Будем искать решение $(\ref{eq10_1})$ в виде
\[y(x) = C_1(x)\varphi_1(x) + ... + C_n(x)\varphi_n(x)\]

Тогда получается, что решение эквивалентной системы будем искать в виде

\begin{equation}
\vec{v}(x) = 
	\begin{Vmatrix}
  		v_1(x)\\
  		...\\
  		v_n(x)
	\end{Vmatrix} = C_1(x)
		\begin{Vmatrix}
  			\varphi_1(x)\\
  			...\\
  			\varphi_1^{(n-1)}(x)
		\end{Vmatrix} + ... + C_n(x)
			\begin{Vmatrix}
  				\varphi_n(x)\\
  				...\\
  				\varphi_n^{(n-1)}(x)
			\end{Vmatrix}
\end{equation}

Рассмотрим функцию $v_k(x) = C_1(x)\varphi_1^{(k-1)} + ... + C_n(x)\varphi_n^{(k-1)}$. Продифференцируем эту функицю по $x$:
\begin{equation}
\forall k =\overline{1, n-1} \hookrightarrow \dot{v_k}(x) = \dot{C_1}(x)\varphi_1^{(k-1)} + ... + \dot{C_n}(x)\varphi_n^{(k-1)} + C_1(x)\varphi_1^{(k)} + ... + C_n(x)\varphi_n^{(k)}
\end{equation}

С другой стороны $\dot{v_k}(x) = v_{k+1} = C_1(x)\varphi_1^{(k)} + ... + C_n(x)\varphi_n^{(k)}$. Тогда получаем
\begin{equation}
\dot{v_k}(x) = C_1(x)\varphi_1^{(k)} + ... + C_n(x)\varphi_n^{(k)} = \dot{C_1}(x)\varphi_1^{(k-1)} + ... + \dot{C_n}(x)\varphi_n^{(k-1)} + C_1(x)\varphi_1^{(k)} + ... + C_n(x)\varphi_n^{(k)}
\end{equation}
\begin{equation}
\forall k =\overline{1, n-1} \hookrightarrow \dot{C_1}(x)\varphi_1^{(k-1)} + ... + \dot{C_n}(x)\varphi_n^{(k-1)} = 0
\end{equation}
\begin{eqnarray*}
k = n: ~\dot{v_n}(x) = \dot{C_1}(x)\varphi_1^{(n-1)} + ... + \dot{C_n}(x)\varphi_n^{(n-1)} + C_1(x)\varphi_1^{(n)} + ... + C_n(x)\varphi_n^{(n)} = \\ = f(x) - a_1(x)\left(C_1(x)\varphi_1^{(n-1)} + ... + C_n(x)\varphi_n^{(n-1)}\right) - ... - a_n(x)\left(C_1(x)\varphi_1 + ... + C_n(x)\varphi_n\right)
\end{eqnarray*}

\begin{eqnarray*}
\dot{C_1}(x)\varphi_1^{(n-1)} + ... + \dot{C_n}(x)\varphi_n^{(n-1)} + C_1(x)\left(\varphi_1^{(n)} + a_1(x)\varphi_1^{(n-1)} + ... + a_n(x)\varphi_1\right) + ... + \\ + C_n(x)\left(\varphi_n^{(n)} + a_1(x)\varphi_n^{(n-1)} + ... + a_n(x)\varphi_n\right) = f(x)
\end{eqnarray*}

Из уравнения $(\ref{eq10_2})$ следует что выражения в скобках равны нулю, тогда получим

\[k = n: ~\dot{C_1}(x)\varphi_1^{(n-1)} + ... + \dot{C_n}(x)\varphi_n^{(n-1)} = f(x)\]

Т.е. мы получили следующую систему уравнений:
\begin{equation}\label{eq10_4}
 \begin{cases}
   \dot{C_1}(x)\varphi_1 + ... + \dot{C_n}(x)\varphi_n = 0, 
   \\
   ...
   \\
   \dot{C_1}(x)\varphi_1^{(n-2)} + ... + \dot{C_n}(x)\varphi_n^{(n-2)} = 0,
   \\
   \dot{C_1}(x)\varphi_1^{(n-1)} + ... + \dot{C_n}(x)\varphi_n^{(n-1)} = f(x).
 \end{cases}
\end{equation}
Система $(\ref{eq10_4})$ это линейная система для определения $\dot{C_1}, ~..., ~\dot{C_n}$.
Определитель этой системы $\Delta = W(x) \neq 0$, а значит система разрешима единственным образом.