\documentclass[a4paper, 12pt]{article}
\documentclass[a4paper, 12pt]{article}

%%% Работа с русским языком
\usepackage{cmap}					% поиск в PDF
\usepackage{mathtext} 				% русские буквы в формулах
\usepackage[T2A]{fontenc}			% кодировка
\usepackage[utf8]{inputenc}			% кодировка исходного текста
\usepackage[russian]{babel}	% локализация и переносы

%%% Дополнительная работа с математикой
\usepackage{amsmath,amsfonts,amssymb,amsthm,mathtools} % AMS
\usepackage{icomma} % "Умная" запятая: $0,2$ --- число, $0, 2$ --- перечисление

%% Номера формул
%\mathtoolsset{showonlyrefs=true} % Показывать номера только у тех формул, на которые есть \eqref{} в тексте.
%\usepackage{leqno} % Немуреация формул слева

%% Шрифты
\usepackage{euscript}	 % Шрифт Евклид
\usepackage{mathrsfs} % Красивый матшрифт

%%% Свои команды
\DeclareMathOperator{\sgn}{\mathop{sgn}}

%% Поля
\usepackage[left=2cm,right=2cm,top=2cm,bottom=2cm,bindingoffset=0cm]{geometry}

%% Русские списки
\usepackage{enumitem}
\makeatletter
\AddEnumerateCounter{\asbuk}{\russian@alph}{щ}
\makeatother

%%% Работа с картинками
\usepackage{graphicx}  % Для вставки рисунков
\graphicspath{{images/}{images2/}}  % папки с картинками
\setlength\fboxsep{3pt} % Отступ рамки \fbox{} от рисунка
\setlength\fboxrule{1pt} % Толщина линий рамки \fbox{}
\usepackage{wrapfig} % Обтекание рисунков и таблиц текстом

%%% Работа с таблицами
\usepackage{array,tabularx,tabulary,booktabs} % Дополнительная работа с таблицами
\usepackage{longtable}  % Длинные таблицы
\usepackage{multirow} % Слияние строк в таблице

%% Красная строка
\setlength{\parindent}{2em}

%% Интервалы
\linespread{1}
\usepackage{multirow}

%% TikZ
\usepackage{tikz}
\usetikzlibrary{graphs,graphs.standard}

%% Верхний колонтитул
\usepackage{fancyhdr}
\pagestyle{fancy}

%% Перенос знаков в формулах (по Львовскому)
\newcommand*{\hm}[1]{#1\nobreak\discretionary{}
	{\hbox{$\mathsurround=0pt #1$}}{}}

%% дополнения
\usepackage{float} %Добавляет возможность работы с командой [H] которая улучшает расположение на странице
\usepackage{gensymb} %Красивые градусы
\usepackage{caption} % Пакет для подписей к рисункам, в частности, для работы caption*

% подключаем hyperref (для ссылок внутри  pdf)
\usepackage[unicode, pdftex]{hyperref}

%%% Теоремы
\theoremstyle{plain}                    % Это стиль по умолчанию, его можно не переопределять.
\renewcommand\qedsymbol{$\blacksquare$} % переопределение символа завершения доказательства

\newtheorem{theorem}{Теорема}[section] % Теорема (счетчик по секиям)
\newtheorem{proposition}{Утверждение}[section] % Утверждение (счетчик по секиям)
\newtheorem{definition}{Определение}[section] % Определение (счетчик по секиям)
\newtheorem{corollary}{Следствие}[theorem] % Следстиве (счетчик по теоремам)
\newtheorem{problem}{Задача}[section] % Задача (счетчик по секиям)
\newtheorem*{remark}{Примечание} % Примечание (можно переопределить, как Замечание)
\newtheorem{lemma}{Лемма}[section] % Лемма (счетчик по секиям)

\newtheorem{example}{Пример}[section] % Пример
\newtheorem{counterexample}{Контрпример}[section] % Контрпример
\newcommand{\defeq}{\stackrel{def}{=}} % по определению
\newcommand{\defarr}{\stackrel{def}{\Rightarrow}} % следует из определения

\makeatletter
\newcommand{\eqnum}{\refstepcounter{equation}\textup{\tagform@{\theequation}}}
\makeatother % создание метки и нумерация формулы одновременно

\newcommand{\deflimk}{\lim\limits_{k\rightarrow \infty}} % лимит при k -> бесконечности
\DeclareMathOperator{\Tr}{trace} % след матрицы
\DeclareMathOperator{\diverg}{div} % определение нормально выглядещей дивергенции
\newcommand{\bignu}{\nu} %Большое ню, если кто-то потом узнает, как его писать

%% типы фазовых траекторий, а именно
% положение равновесия
% замкнутая траектория
% траектория без самопересечений
%% должны быть рассмотрены ранее

\begin{document}
    \subsection{Классификация положений равновесия линейной
    автономной системы второго порядка. Характер поведения фазовых траекторий в окрестности положения равновесия двумерной автономной нелинейной системы. Теорема о выпрямлении траекторий.}

    Рассмотрим систему уравнений

    \begin{equation}
      \begin{cases}
        \frac{d x}{d t} = f_1 (x, y) \\
        \frac{d y}{d t} = f_2 (x, y) \\
      \end{cases} 
    \end{equation}

    Пусть $M_0(x_0, y_0)$ -- положение равновесия данной системы, т. е. выполнено:
    $\begin{cases}
      \frac{d x}{d t}(x_0, y_0) = 0 \\
      \frac{d y}{d t}(x_0, y_0) = 0
    \end{cases}$
    Тогда, мы можем формально линеаризовать систему, используя известные методы:

    \begin{equation}
      \begin{cases}
        \frac{d x}{d t} = \frac{\partial f_1}{\partial x} (x - x_0) + \frac{\partial f_1}{\partial y} (y - y_0) + o(\rho) \\
        \frac{d y}{d t} = \frac{\partial f_2}{\partial x} (x - x_0) + \frac{\partial f_2}{\partial y} (y - y_0) + o(\rho) \\        
      \end{cases}
    \end{equation}

    где $\rho = \sqrt{(x - x_0)^2 + (y - y_0)^2}$. В итоге, стандартной заменой $x = \overline{x} + x_0$ и $y = \overline{y} + y_0$ приводим систему к линейному виду.

    \begin{equation}
      \begin{cases}
        \frac{d \overline{x}}{d t} = \alpha_{11} \overline{x} + \alpha_{12} \overline{y} \\
        \frac{d \overline{y}}{d t} = \alpha_{21} \overline{x} + \alpha_{22} \overline{y} \\        
      \end{cases}
    \end{equation}

    С этого момента, мы будем изучать виды фазвых траекторий и их поведение в окрестности положения равновесия для систем вида:

    \begin{equation} \label{eq:base_system}
      \begin{cases}
        \frac{d x}{d t} = a_{11} x + a_{12} y \\
        \frac{d y}{d t} = a_{21} x + a_{22} y \\        
      \end{cases}
    \end{equation}

    с положением равновесия в точке $M_0(0, 0)$.

    Рассмотрим автономную однородную систему линейных ДУ \eqref{eq:base_system} и введем матрицу системы:
    
    \begin{equation}
      A = 
      \begin{pmatrix}
        a_{11} ~~ a_{12} \\
        a_{21} ~~ a_{22} \\
      \end{pmatrix}
    \end{equation}

    Получим собственные значения этой матрицы:

    \begin{equation}
      \begin{vmatrix}
        a_{11} - \lambda ~~ a_{12} \\
        a_{21} ~~ a_{22} - \lambda \\
      \end{vmatrix} = 
      \lambda^2 - \Tr A \cdot \lambda - \det A = 0 \Rightarrow
    \end{equation}

    \[ \lambda = \frac{\Tr A \pm \sqrt{\Tr^2 A - 4 \det A}}{2} \]

    Фазовый портрет системы зависит от собственных значений матрицы $A$.

    \begin{enumerate}
      \item Собственные значения  $\lambda_1, ~ \lambda_2 ~ \in ~ \mathbb{R}$ (или $\Tr^2 A - 4 \det A > 0$)
      
      Тогда, в базисе собственных значений матрица $A$ примет вид:
      $\overline{A} = 
      \begin{pmatrix}
        \lambda_1 ~~ 0 \\
        0 ~~ \lambda_2 \\
      \end{pmatrix}$

      система \eqref{eq:base_system} будет иметь вид:
      $\begin{cases}
        \frac{d x}{d t} = \lambda_1 x \\
        \frac{d y}{d t} = \lambda_2 y \\        
      \end{cases}$

      и решения данной системы в базисе собственных векторов:
      $\begin{cases}
        x(t) = c_1 e^{\lambda_1 t} \\
        y(t) = c_2 e^{\lambda_2 t} \\
      \end{cases}$

      Решение системы в исходном базисе:
      $\begin{cases}
        x(t) = c_1 e^{\lambda_1 t} h_1 \\
        y(t) = c_2 e^{\lambda_2 t} h_2 \\
      \end{cases}$

      где $h_1, ~ h_2$ -- собственные векторы матрицы $A$.

      Рассмотрим фазовые портреты.

      \begin{enumerate}
        \item $\lambda_1 < 0, ~~ \lambda_2 < 0$ и $|\lambda_1| < |\lambda_2|$
        
        Заметим прежде всего, что при $c_1 \neq 0, ~ c_2 = 0$ и при $c_1 = 0, ~ c_2 \neq 0$ мы получаем прямые линии с направляющими векторами $h_1$ и $h_2$. Поэтому вектора $h_1$ и $h_2$ являются решениями системы.

        Теперь, рассмотрим, что будет при $c_1 \neq 0$ и $c_2 \neq 0$. Из 
        $\displaystyle 
        \begin{cases}
          x(t) = c_1 e^{\lambda_1 t} \\
          y(t) = c_2 e^{\lambda_2 t} \\
        \end{cases} \Rightarrow t = \frac{1}{\lambda_1} \ln{\frac{x}{c_1}}$ подставляем выражение для $y$ и получаем \textbf{в базисе собственных векторов} $\displaystyle y = c |x|^{\frac{\lambda_2}{\lambda_2}} = c |x|^r$, где $\displaystyle r = \frac{\lambda_2}{\lambda_1} > 0$.

        Таким образом мы приходим к выводу, что фазовые трактории в данном случае -- есть параболы (с показателем $r > 0$), причем при $t \rightarrow 0$ фазовые траектории стремяться к положению равновесия.

        \begin{definition}
          Положение равновесия, при котором сосбтвенные значения матрицы $A$ одного знака и фазовые трактории направлены к положению равновесия называются \textbf{устойчивым узлом}.
        \end{definition}

        \begin{remark}
          В случае, когда положение равновесия является узлом, фазовые траектории касаются оси с меньшим по модулю собственным числом.
        \end{remark}

      \end{enumerate}

    \end{enumerate}

\end{document}