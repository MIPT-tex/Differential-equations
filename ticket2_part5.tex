\documentclass[a4paper, 12pt]{article}

%%% Работа с русским языком
\usepackage{cmap}					% поиск в PDF
\usepackage{mathtext} 				% русские буквы в формулах
\usepackage[T2A]{fontenc}			% кодировка
\usepackage[utf8]{inputenc}			% кодировка исходного текста
\usepackage[russian]{babel}	% локализация и переносы

%%% Дополнительная работа с математикой
\usepackage{amsmath,amsfonts,amssymb,amsthm,mathtools} % AMS
\usepackage{icomma} % "Умная" запятая: $0,2$ --- число, $0, 2$ --- перечисление

%% Номера формул
%\mathtoolsset{showonlyrefs=true} % Показывать номера только у тех формул, на которые есть \eqref{} в тексте.
%\usepackage{leqno} % Немуреация формул слева

%% Шрифты
\usepackage{euscript}	 % Шрифт Евклид
\usepackage{mathrsfs} % Красивый матшрифт

%%% Свои команды
\DeclareMathOperator{\sgn}{\mathop{sgn}}

%% Поля
\usepackage[left=2cm,right=2cm,top=2cm,bottom=2cm,bindingoffset=0cm]{geometry}

%% Русские списки
\usepackage{enumitem}
\makeatletter
\AddEnumerateCounter{\asbuk}{\russian@alph}{щ}
\makeatother

%%% Работа с картинками
\usepackage{graphicx}  % Для вставки рисунков
\graphicspath{{images/}{images2/}}  % папки с картинками
\setlength\fboxsep{3pt} % Отступ рамки \fbox{} от рисунка
\setlength\fboxrule{1pt} % Толщина линий рамки \fbox{}
\usepackage{wrapfig} % Обтекание рисунков и таблиц текстом

%%% Работа с таблицами
\usepackage{array,tabularx,tabulary,booktabs} % Дополнительная работа с таблицами
\usepackage{longtable}  % Длинные таблицы
\usepackage{multirow} % Слияние строк в таблице

%% Красная строка
\setlength{\parindent}{2em}

%% Интервалы
\linespread{1}
\usepackage{multirow}

%% TikZ
\usepackage{tikz}
\usetikzlibrary{graphs,graphs.standard}

%% Верхний колонтитул
\usepackage{fancyhdr}
\pagestyle{fancy}

%% Перенос знаков в формулах (по Львовскому)
\newcommand*{\hm}[1]{#1\nobreak\discretionary{}
	{\hbox{$\mathsurround=0pt #1$}}{}}

%% дополнения
\usepackage{float} %Добавляет возможность работы с командой [H] которая улучшает расположение на странице
\usepackage{gensymb} %Красивые градусы
\usepackage{caption} % Пакет для подписей к рисункам, в частности, для работы caption*

% подключаем hyperref (для ссылок внутри  pdf)
\usepackage[unicode, pdftex]{hyperref}

%%% Теоремы
\theoremstyle{plain}                    % Это стиль по умолчанию, его можно не переопределять.
\renewcommand\qedsymbol{$\blacksquare$} % переопределение символа завершения доказательства

\newtheorem{theorem}{Теорема}[section] % Теорема (счетчик по секиям)
\newtheorem{proposition}{Утверждение}[section] % Утверждение (счетчик по секиям)
\newtheorem{definition}{Определение}[section] % Определение (счетчик по секиям)
\newtheorem{corollary}{Следствие}[theorem] % Следстиве (счетчик по теоремам)
\newtheorem{problem}{Задача}[section] % Задача (счетчик по секиям)
\newtheorem*{remark}{Примечание} % Примечание (можно переопределить, как Замечание)
\newtheorem{lemma}{Лемма}[section] % Лемма (счетчик по секиям)

\newtheorem{example}{Пример}[section] % Пример
\newtheorem{counterexample}{Контрпример}[section] % Контрпример
\newcommand{\defeq}{\stackrel{def}{=}} % по определению
\newcommand{\defarr}{\stackrel{def}{\Rightarrow}} % следует из определения
\DeclareMathOperator{\diverg}{div} % определение нормально выглядещей дивергенции

\begin{document}
    \section*{Билет 2, часть 5}
    \subsection*{Теоремы о продолжении решения для нормальной системы дифференциальных уравнений.}

    Теоремы Коши носят существенно локальный характер. Решение и единственность задачи Коши будет существовать на отрезке Пиано. Теперь сделаем отход от единственности и докажем, чтo $\Vec{\varphi}(t)$ и $\Vec{\psi}(t)$ есть решение задачи Коши, то они будут совпадать на промежутке, где они оба определены (отход от локальности).

    %% (1) и (2) опредленно в пред части билета
	\begin{theorem}
		Пусть $\Vec{\varphi}(t)$ решение $(1) \wedge (2)$ определенно на $[a, b]$, a  $\Vec{\psi}(t)$ решение $(1) \wedge (2)$ определенно на $[c, d]$.
		Тогда $\Vec{\varphi}(t) \equiv \Vec{\psi}(t)$ на $[r_1, r_2] = [a, b] \cap [c, d]$.
	\end{theorem}
	\begin{proof}
        От противного: $\exists t^* \in [r_1, r_2]$, где $\Vec{\varphi}(t^*) \neq \Vec{\psi}(t^*)$, тогда $t^* \neq t_0$ и предположим, что $t^* > t_0$. 
        Рассмотрим мн-во $N$ точек такое, что $t \in [r_1, r_2]$ и $\Vec{\varphi}(t) = \Vec{\psi}(t)$. \\
        Покажем, что \underline{мн-во замкнуто}: \\
        Рассмотрим сходящуюся послед-сть $t_1 \ldots t_n \in \mathbb{N}$, $\displaystyle \lim_{n\to\infty} t_n = \overline{t}$. Нужно показать, что $\overline{t} \in \mathbb{N}$:\\
        Рассмотрим $\displaystyle \lim_{n\to\infty} \Vec{\varphi}(t_n) = \lim_{n\to\infty} \Vec{\psi}(t_n)$ (равны по выбору множества $N$). И из непрерывности выбранных функций получаем, что $\displaystyle \lim_{n\to\infty} \Vec{\varphi}(t_n) = \lim_{n\to\infty} \Vec{\psi}(t_n) = \Vec{\varphi}(\overline{t}) = \Vec{\psi}(\overline{t})$ $\Rightarrow$ замкнутость. \\
        Из замкнутости и ограниченности мн-ва $N$ $\Rightarrow$ $\exists \widetilde{t} = sup\ N$, $\widetilde{t} \in N$. Мы пришли к противоречию, а именно $t^*$ по начальному предположению должна быть точной верхней гранью.
    \end{proof}
    
    \begin{definition} 
        $\Vec{\varphi}(t)$ определена на $\langle a, b\rangle$ и решение $(1) \wedge (2)$, если $\exists \Vec{\psi}(t)$ на $\langle a, b_1\rangle \supset \langle a, b\rangle$, и решение $(1) \wedge (2)$ и $\Vec{\varphi}(t) \equiv \Vec{\psi}(t)$ на $\langle a, b\rangle$, тогда $\Vec{\varphi}(t)$ называется продолжаемым вправо, а $\Vec{\psi}(t)$ продолжением решения $\Vec{\varphi}(t)$ задачи Коши
    \end{definition}
    
    \begin{definition} 
        Решение, которое нельзя продолжить ни вправо, ни влево называется непродолжаемым решением
    \end{definition}
    
    \begin{remark}
        По сути данная теорема является усилением задачи Коши. Вместо отрезка Пиано мы получили, что решение задачи Коши может быть продолжено на промежуток, где они оба определены.
    \end{remark}
    
    \begin{theorem}
		Пусть имеется задача Коши $(1) \wedge (2)$ и $\vec{f}(t, \vec{x}), \dfrac{\partial f^i}{\partial x_j}, i, j = \overline{1, n}$ непрерывны в $\Omega$ с $\mathbb{R}^{n+1}$. Тогда $\forall (t_0, \vec{x_0}) \in \Omega \ \exists!$ непродолжаемое решение задачи $(1) \wedge (2)$.
	\end{theorem}
	\begin{proof}
	    Рассмотрим множество решений задач Коши $(1) \wedge (2)$. Каждое решение задачи определенно на промежутке $\langle R_1, R_2 \rangle$, тогда пусть $T_1 = inf\ R_1, T_2 = sup\ R_2$. Построим решение задачи $(1) \wedge (2)$ на $(T_1, T_2)$:\\
        Выберем $t^* > t_0$, тогда $\exists\ \vec{\psi}(t)$, чей промежуток содержит $t^*$ (в силу выбора промкежутка $(T_1, T_2)$). Положим $\vec{\varphi}(t^*) \defeq \vec{\psi}(t^*)$. Покажем, что так можем сделать, что значение $\vec{\varphi}(t^*)$ не зависит от выбора $\vec{\psi}(t)$:\\
        Пусть $\displaystyle \vec{\hat{\psi}}(t)$ решение задачи Коши $(1) \wedge (2)$ содержащее $t^*$, тогда $\vec{\hat{\psi}}(t^*) = \vec{\psi}(t^*)$ из теоремы сущ. и единст. решения задачи Коши (будут совпадать на промежутке, где они определены и при этом $t^*$ принадлежит этому промежутку).\\ 
        Построение вниз проводится аналогично. И так, $\vec{\varphi}(t)$ решение $(1) \wedge (2)$ на $T_1 < t < T_2$. Это решение \underline{является продолжением любого из множества решений} задачи Коши. Допустим, $\vec{\widetilde{\varphi}}(t)$ решение $(1) \wedge (2)$ на $r_1 \leq t \leq r_2$ и $T_1 \leq r_1 \leq r_2 \leq T_2$ $\Rightarrow$ $\vec{\widetilde{\varphi}}(t) = \vec{\varphi}(t)$ (продолжение решения по доказанной выше теоремы). \\Покажем, что $\vec{\psi}(t)$ является \underline{непродолжаемым решением} $(1) \wedge (2)$:
        Допустим, что имеется ещё  одно решение $\vec{\chi}(t)$, определённое на $(\gamma_1; \gamma_2)$ и оно является продолжением $\vec{\varphi}(t)$. Тогда, либо $\gamma_1 < T_1$, либо $\gamma_2 > T_2$, что невозможно, т.к. $T_1 = inf\ R_1, T_2 = sup\ R_2$ по построению. \\
        Покажем, что непродолжаемое решение $\vec{\varphi}(t)$ является \underline{единственным}:\\
        От противного, пусть $\exists$ $\vec{\varphi}(t)$ непродолжаемое решение на $(T_1, T_2)$ и $\vec{\psi}(t)$ на $(\widetilde{T_1}, \widetilde{T_2})$. Для определённости $\widetilde{T_1} < T_1$, тогда рассмотрим такое решение
        $\vec{\chi}(t)$ = 
        $\left[ 
            \begin{gathered} 
                \vec{\psi}(t)\ \text{на}\ (\widetilde{T_1}, T_1), \\ 
                \vec{\varphi}(t)\ \text{на}\ (T_1, T_2); \\ 
            \end{gathered} 
        \right.$
        $\Rightarrow$ $\vec{\varphi}(t)$  -- продолжение $\vec{\psi}(t)$, противоречие. Аналогично строя остальные решения получаем, что $\vec{\varphi}(t) = \vec{\psi}(t)$ 
	\end{proof}
	
	\begin{remark}
	    В теореме не скаазано, как определить $T_1$ и $T_2$. Есси усилить условия теоремы, а именно $\Omega$ есть ограниченн ая область, то любое непродолжаемое решение выходит на границу этой области.\\
	    Из этих утверждений следует, что если под интегральной кривой понимать график непродолжаемого решения, то через каждую точку $(x_0, y_0) \in \Omega$ проходит только одна кривая.
	\end{remark}
	
\end{document}
